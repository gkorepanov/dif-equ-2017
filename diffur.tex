\documentclass{urticle}
\usepackage{extarrows,pgffor}

\DeclareMathOperator{\sign}{sign}

\relpenalty=10000
\def\?#1{#1\nobreak\discretionary{}{\hbox{$\mathsurround=0pt #1$}}{}}
\renewcommand{\strut}{\rule[-.15\baselineskip]{0pt}{\baselineskip}}

\newcommand{\raf}[1]{{#1} $\boldsymbol\rightarrow$ }
\newcommand{\scheme}[1]{%
\hyphenpenalty=10000
\begin{center}
	\noindent\fbox{%
    \parbox{0.8\textwidth}{%
    #1
        }%
}
\end{center}
\hyphenpenalty=50
}

\newcommand{\prim}[1]{\textbf{Примечание.} #1}
\newcommand{\name}[1]{\underline{\textsl{#1\strut}}}

\newcommand{\num}[2]{\textbf{#1.} (#2)}
\begin{document}
\begin{center}

\section{by olga}

\reflectbox{\rotatebox[origin=c]{180}{!}} 
Hi!

\section*{Адекватные доказательства теорем из курса дифференциальных уравнений}
В заметке собраны ссылки на наиболее понятные, простые, но вместе с тем достаточно полные доказательства теорем курса дифференциальных уравнений. Нумерация совпадает с нумерацией экзаменационных билетов в программе.

Также даны некоторые замечания по вопросам, не освещенным в указанных учебниках. Приведены планы доказательств.
\end{center}

{
	\num{17}{Автономные системы}
	\name{Филиппов, Арнольд}
	
	Рассматривается поведение автономной системы в окрестности неособой точки, а также классификация ФТ (фазовых траекторий).
	
	\prim{У Арнольда (стр.~65-66 указанные свойства доказаны несколько проще/понятней.}
	
	\prim{\underline{Теорема о выпрямлении траекторий:} в  достаточно малой окрестности любой неособой точки  векторное поле диффеоморфно постоянному полю $\vec{E}$.}
	
	\scheme{
		\raf{Фаз. пространство}%
		\raf{$x(t+c)$~--- решение}%
		\raf{ФТ либо совпадают, либо не $\cap$, т.к. $y(t+t_2-t_1) \equiv x(t)$}%
		\raf{Периодичность}%
		\raf{Наименьший период}%
		\raf{Отсутствие самопересечений}
			{Классификация: ПР, (не)замкнутая без самопересечений}	
	}
}

{
	\num{18}{Фазовый портрет линейной системы $n=2$}
	\name{$\forall$}
	
	Во всех учебниках написано понятно.
	Записываем матрицу в жордановом базисе и анализируем решения при различных $\lambda$.	
	
	\prim{\underline{Теорема о кач. эквивалентности:} если линеаризованная система~--- простая, и ПР~--- не центр, то в некоторой окрестности нелинейная и линеаризованная линейная системы гомеоморфны.}
	\scheme{
		\raf{Жорданов базис}
		\raf{(Не)устойчивый узел ($\sign\lambda_1 = \sign\lambda_2$)}
		\raf{Седло ($\sign\lambda_1 \neq \sign\lambda_2$)}
		\raf{Дикритический узел ($\lambda_1 = \lambda_2 \neq 0$, 2 с.в.)}
		\raf{Вырожденный узел ($\lambda_1 = \lambda_2 \neq 0$, 1 с.в., 1 п.в.)}
		\raf{Фокус, центр ($\lambda \in \mathbb{Z}$)}
		    {Вырожденная матрица~--- всякая фигня}

	}
}

{
	\num{13/22}{Теорема Коши}
	\name{Эльсгольц}
	
	Используется \textsf{принцип сжатых отображений}, доказательство которого~(стр.~48-49) практически очевидно, дальше вручную проверяется, что оператор $A[y]$ (интегральной формы диф. уравнения) является сжимающим.
	
	\scheme{
		\raf{Принцип сжимающих отображений}%
		\raf{Замена диф.~уравнения интегральным}%
		\raf{Введение оператора $A[y]$}%
		\raf{Условие Липшица}%
		\raf{Проверка, что $A[y]$~--- сжимающий}%
		    {Обобщение на случай систем}	
	}
}

{
	\num{19}{Первые интегралы}
	\name{Романко, Филиппов}
	
	В Филиппове (стр. 213) есть несколько слов о \textsf{критерии первого интеграла} ($u$~--- п.и. $\?\Longleftrightarrow \dot{u}(t)_f = 0$) и о \textsf{понижении порядка} системы при наличии нетривиального первого интеграла.
	
	В Романко есть то же самое, но нудно доказанное на целую страницу~(стр.~252, стр.~257).
	
	\scheme{
		\raf{Определение}%
		\raf{Критерий}%
		\raf{Разрешение $u(x) = C$ относительно $x_i$}%
		\raf{Подстановка $x_i$ в систему}%
		    {Проверка обратимости такой замены}	
	}
}


{
	\num{20}{$\exists$ $n-1$ первых интегралов}
	\name{Арнольд}
	
	Эта теорема~--- прямое следствие теоремы о выпрямлении траекторий, т.к. преобразование координат инвариантно относительно свойства <<быть первым интегралом>> (стр.~72).


}


{
	\num{23/24}{ФСР, вронскиан}
	\name{Филиппов}
	
	Доказывается все для \textsf{однородной системы}~(стр.~67-78), затем  почти очевидным образом переносится на \textsf{линейные уравнения}~(стр.~81-86).
	
	\scheme{
	        \raf{Линейная независимость}%
	        \raf{Вронскиан}%
	        \raf{Фунд.~система решений}%
	        \raf{Дифференцирование детерминанта}%
	        \raf{Формула Лиувилля}%
	            {Замена переменных (переход от системы к лин.~уравнению)}
	}
}

{
	\num{25}{Вариация постоянных}
	\name{Филиппов}
	
	Аналогично предыдущему. Доказывается в одну строчку~(стр.~79) для \textsf{систем}, затем заменой переменных~(стр.~90) для \textsf{линейных уравнений}.
	
	\prim{Система уравнений для вариации постоянной в случае линейного уравнения получается из явной записи матричного равенства $X(t)c'(t)=f_0(t)$.}
	
	\scheme{
	        \raf{Вариация постоянных ($c = c(t)$)}%
	        \raf{Дифференцирование общего решения}%
	        \raf{Подстановка в неоднородное уравнение}%
	        \raf{Окончательная формула через обратную матрицу}%
	            {Замена переменных (переход от системы к лин.~уравнению)}
	}
}

{
	\num{26}{Теорема Штурма и следствия}
	\name{Романко, Филиппов}
	
	Доказательство Романко почти повторяет Филиппова, но переходы освещены более подробно~(стр.~193-197). Кроме того, в \textsf{Филиппове есть не все следствия.} Но следствие о расстоянии между нулями уравнения $y''+q(x)y=0$ с $m^2 \leq y(x)\leq M^2$ понятнее в Филиппове~(стр.~113).
	
	\scheme{
	        \raf{Замена переменного}%
	        \raf{Общий вид $y''+q(x) y = 0$}%
	        \raf{Лемма о простых нулях ($y(x_0) = 0 \Rightarrow y'(x_0) \neq 0$)}%
	        \raf{Без~о.о. $y > 0, z > 0$}%
	        \raf{Домножение исходных уравнений на $y$ и $z$, вычитание полученных равенств и интегрирование по $x$}%
	        	{Теорема Штурма и несколько очевидных следствий, полученных из разных оценок с разными $Q(x)$} 
	}
}

{
	\num{27}{Устойчивость по Ляпунову}
	\name{Романко, Филиппов}
	
	Определение \textsf{устойчивости по Ляпунову} и \textsf{асимптотической устойчивости} дано в Романко (стр.~241-242).
	Достаточные условия асимпт. устойчивости приведены в теореме в Филиппове~(стр.165), нужна только 1-ая часть доказательства.
	
	\prim{Устойчивость по Ляпунову необходимо требует \textsf{продолжимости решений бесконечно вправо} в малой окрестности положения равновесия (см.~билет~14).}
	
	\prim{Теоремы Ляпунова об (асимпт.) устойчивости, теорема Четаева и теорема об устойчивости по лин. приближению~--- без доказательства}
	
	\scheme{
		\raf{Устойчивость по Ляпунову}%
		\raf{Асимптотическая устойчивость}%
		\raf{Общий вид решения линейной системы}%
		\raf{ФСР~$X(0) = E$ (см.~$e^{At}$~(билет~9))}%
		\raf{Ограниченность $||X(t)|| < M$}
			{Оценка сверху с $\delta = \varepsilon /M$}
			
	
	}
}


\end{document}
