\documentclass{urticle}
\usepackage{extarrows,pgffor}

\newcommand{\raf}[1]{#1 $\rightarrow$ }
\newcommand{\scheme}[1]{\begin{center}
	\noindent\fbox{%
    \parbox{0.8\textwidth}{%
    #1
        }%
}
\end{center}
}

\newcommand{\name}[1]{\underline{\textsl{#1}}}

\newcommand{\num}[1]{\textbf{#1.}}
\begin{document}
\begin{center}
\section*{Адекватные доказательства теорем по дифференциальным уравнениям}
\end{center}

{
	\num{22}
	\name{Эльсгольц}
	
	Используется \textsf{принцип сжатых отображений}, доказательство которого (стр. 48-49) практически очевидно, дальше вручную проверяется, что оператор $A[y]$ (интегральной формы диф. уравнения) является сжимающим.
	
	\scheme{
		\raf{Принцип сжимающих отображений}
		\raf{замена диф.~уравнения интегральным}
		\raf{введение оператора $A[y]$}
		\raf{условие Липшица}
		\raf{проверка, что $A[y]$~--- сжимающий}
		    {ручное обобщение на случай систем}	
	}
}

{
	\num{23/24}
	\name{Филиппов}
	
	Доказывается все для \textsf{однородной системы} (стр.~67-78), затем  почти очевидным образом переносится на \textsf{линейные уравнения} (стр.~81-86).
	
	\scheme{
	        \raf{Линейная независимость}
	        \raf{вронскиан}
	        \raf{фунд.~система решений}
	        \raf{дифференцирование детерминанта}
	        \raf{формула Лиувилля}
	            {замена переменных (переход от системы к лин.~уравнению)}
	}
}

{
	\num{25}
	\name{Филиппов}
	
	Аналогично предыдущему. Доказывается в одну строчку (стр. 79) для \textsf{систем}, затем заменой переменных (стр. 90) для \textsf{линейных уравнений}.
	
	\scheme{
	        \raf{Вариация постоянных ($c = c(t)$}
	        \raf{дифференцирование общего решения}
	        \raf{подстановка в неоднородное уравнение}
	        \raf{окончательная формула через обратную матрицу}
	            {замена переменных (переход от системы к лин.~уравнению)}
	}
}


\end{document}
