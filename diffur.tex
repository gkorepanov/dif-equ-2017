\documentclass{urticle}
\usepackage{extarrows,pgffor}

\newcommand{\raf}[1]{{#1} $\rightarrow$}
\newcommand{\scheme}[1]{\begin{center}
	\noindent\fbox{%
    \parbox{0.8\textwidth}{%
    #1
        }%
}
\end{center}
}

\newcommand{\name}[1]{\underline{\textsl{#1}}}

\newcommand{\num}[1]{\textbf{#1.}}
\begin{document}
\begin{center}
\section*{Адекватные доказательства теорем по дифференциальным уравнениям}
\end{center}

{
	\num{22}
	\name{Эльсгольц}
	
	Используется \textsf{принцип сжатых отображений}, доказательство которого (стр.~48-49) практически очевидно, дальше вручную проверяется, что оператор $A[y]$ (интегральной формы диф. уравнения) является сжимающим.
	
	\scheme{
		\raf{Принцип сжимающих отображений}
		\raf{Замена диф.~уравнения интегральным}
		\raf{Введение оператора $A[y]$}
		\raf{Условие Липшица}
		\raf{Проверка, что $A[y]$~--- сжимающий}
		    {Ручное обобщение на случай систем}	
	}
}

{
	\num{23/24}
	\name{Филиппов}
	
	Доказывается все для \textsf{однородной системы} (стр.~67-78), затем  почти очевидным образом переносится на \textsf{линейные уравнения} (стр.~81-86).
	
	\scheme{
	        \raf{Линейная независимость}
	        \raf{Вронскиан}
	        \raf{Фунд.~система решений}
	        \raf{Дифференцирование детерминанта}
	        \raf{Формула Лиувилля}
	            {Замена переменных (переход от системы к лин.~уравнению)}
	}
}

{
	\num{25}
	\name{Филиппов}
	
	Аналогично предыдущему. Доказывается в одну строчку (стр.~79) для \textsf{систем}, затем заменой переменных (стр.~90) для \textsf{линейных уравнений}.
	
	\scheme{
	        \raf{Вариация постоянных ($c = c(t)$}
	        \raf{Дифференцирование общего решения}
	        \raf{Подстановка в неоднородное уравнение}
	        \raf{Окончательная формула через обратную матрицу}
	            {Замена переменных (переход от системы к лин.~уравнению)}
	}
}

{
	\num{26}
	\name{Романко, Филиппов}
	
	Доказательство слово-в-слово повторяет Филиппова, но переходы освещены более подробно (стр.~193-197). Кроме того, в \textsf{Филиппове есть не все следствия.} Но следствие о расстоянии между нулями уравнения $y''+q(x)y=0$ с $m^2 \leq y(x)\leq M^2$ понятнее в Филиппове (стр.~113).
	
	\scheme{
	        \raf{Замена переменного}
	        \raf{Общий вид $y''+q(x) y = 0$}
	        \raf{Лемма о простых нулях ($y(x_0) = 0 \Rightarrow y'(x_0) \neq 0$)}
	        \raf{Без~о.о. $y > 0, z > 0$}
	        \raf{Домножение исходных уравнений на $y$ и $z$, вычитание полученных равенств и интегрирование по $x$}
	        	{Теорема Штурма и несколько очевидных следствий, полученных из разных оценок с разными $Q(x)$} 
	}
}


\end{document}
